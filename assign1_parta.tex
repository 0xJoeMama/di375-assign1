\documentclass[draft]{article}
\usepackage{babel}
\usepackage{geometry}
\usepackage{fontspec}
\usepackage{amsmath}
\usepackage{amssymb}
\usepackage{circuitikz}
\setmainfont{NotoSans-Regular}
\geometry{margin=0.5in}
\selectlanguage{greek}

\author{Πλαστήρας Πέτρος}
\date{12 November 2024}
\title{Εργαστήριο Λογικής Σχεδίασης - Εργασία 1}
\begin{document}
\maketitle

\section{Μέρος Α}
\subsection{Ερώτημα 1}
Μία γραμμή internet έχει ταχύτητα 24Mbit/sec.\\
Πόσους μη προσημασμένους αριθμούς των 32bit μπορεί να μεταδώσει σε 35 sec, 
αν επιπλέον εφαρμόζει και parity check (περιττό) για κάθε αριθμό που στέλνει;\\
\textbf{Απάντηση: }\\
Παρατηρούμε αρχικά ότι για κάθε 32bit αριθμό, θα αποστέλονται στην πραγματικότητα 33bit αφού πρέπει να \\
αποσταλείκαι ένα επιπλέον bit για το parity check. Συνεπώς το ερώτημα είναι πόσα πακέτα των 33bit μπορούν να αποσταλούν μέσω της γραμμής σε 35 sec.
Αν $T$ είναι η ταχύτητα της γραμμής και $D$ ο αριθμός των bits που αποστέλονται ανά πακέτο, τότε ισχύει:
\begin{align*}
  D &= 33bits \\
  T &= 24Mbit/sec
\end{align*}

Από την δεύτερη εξίσωση προκύπτει ότι:
\begin{align*}
  T &= 24Mbit/sec \\
  T &= \frac{24 \cdot 1024}{1} \frac{kbit}{sec} = 24 \cdot 1024^2 \frac{bits}{sec}\\
\end{align*}

Συνεπώς στα 35sec μπορούν να μεταφερθούν:
$$
D_t = T \cdot 35sec = 24 \cdot 1024^2 \cdot 35 \ \frac{bits \cdot sec}{sec} = 880,803,840 bits
$$
όπου $D_t$ ο συνολικός όγκος δεδομένων που μπορεί να αποσταλεί σε $35sec$. 

Τελικά, προκειμένου να δούμε πόσοι αριθμοί μπορούν να σταλούν αρκεί να 
διαιρέσουμε το $D_t$ με το μέγεθος του ενός πακέτου($33bits$):

$$
n = \frac{D_t}{D} = \frac{880,803,840 bits}{33 bits} = 26,691,025.\bar{45} \approx 26,691,025
$$

Θα αποσταλθούν συνεπώς 26,691,025 αριθμοί των 32bit με parity check.

\subsection{Ερώτημα 2}
Κάθε μια από τις παρακάτω πράξεις μπορεί να είναι σωστή σε ένα ή
περισσότερα αριθμητικά συστήματα. Προσδιορίστε τις πιθανές βάσεις των
αριθμητικών συστημάτων 

\textbf{Απάντηση: }
Σε όλα τα παρακάτω θεωρώ ότι: $a \in \mathbb{Z} \land a > 1$ όπου $a$ η βάση του αριθμητικού συστήματος.
Γνωρίζουμε ότι σε κάθε αριθμητικό σύστημα το κάθε ψηφίο ενός αριθμού αντιστοιχεί στον αριθμό των φορών που ο αριθμός περιέχει την δύναμη της βάσης που αντιστοιχεί στην στήλη του ψηφίου.
Για παράδειγμα ο αριθμός $598$ στο δεκαδικό σύστημα, γράφεται και ως $5 \cdot 10^2 + 9 \cdot 10^1 + 8 \cdot 10^0$. Το αντίστοιχο ισχύει για όλα τα αριθμητικά συστήματα.
Αν λοιπόν $a$ η βάση του αριθμητικού συστήματος τότε για κάθε παράδειγμα ισχύει:
\subsubsection{α}
\begin{align*}
  \frac{51}{3} &= 15 \\
  \frac{5a + 1}{3} &= 1a + 5 \\
  5a + 1 &= 3(a + 5) \\
  5a + 1 &= 3a + 15 \\
  2a &= 14 \\
  a &= 7
\end{align*}
Συνεπώς το $\frac{51}{3} = 15$ σημαίνει ότι οι πράξεις γίνονται στο αριθμητικό σύστημα με βάση το $7$.

\subsubsection{β}
\begin{align*}
  \frac{44}{4} &= 11\\ 
  \frac{4a + 4}{4} &= a + 1\\
  4a + 4 &= 4a + 4
\end{align*}
Το παραπάνω ισχύει $\forall a \in \mathbb{Z}$. Συνεπώς σε κάθε αριθμητικό σύστημα που μπορεί να χρησιμοποιήσει το ψηφίο $4$ ισχύει ότι $\frac{44}{4} = 11$.
Για να μπορεί σε ένα σύστημα να αναπαρασταθεί το $4$ πρέπει η βάση του να είναι μεγαλύτερη του $4$. Συνεπώς το παραπάνω ισχύει για κάθε σύστημα μετά το πενταδικό.

\end{document}
