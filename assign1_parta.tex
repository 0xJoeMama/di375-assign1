\documentclass[]{article}
\usepackage{geometry}
\usepackage{fontspec}
\usepackage{amsmath}
\usepackage{amssymb}
\usepackage{circuitikz}
\setmainfont{NotoSans}
\geometry{margin=0.5in}

\begin{document}
\author{Πλαστήρας Πέτρος}
\title{Εργαστήριο Λογικής Σχεδίασης - Εργασία 1 - Μέρος Α}
\date{25 Νοεμβρίου 2024}
\maketitle

\section{Ερώτημα 1}
Μία γραμμή internet έχει ταχύτητα 24Mbit/sec.
Πόσους μη προσημασμένους αριθμούς των 32bit μπορεί να \\ μεταδώσει σε 35 sec,
αν επιπλέον εφαρμόζει και parity check (περιττό) για κάθε αριθμό που στέλνει;

\textbf{Απάντηση: }\\
Παρατηρούμε αρχικά ότι για κάθε 32bit αριθμό, θα αποστέλονται στην πραγματικότητα 33bit αφού πρέπει να \\
αποσταλείκαι ένα επιπλέον bit για το parity check. Συνεπώς το ερώτημα είναι πόσα πακέτα των 33bit μπορούν να αποσταλούν μέσω της γραμμής σε 35 sec.
Αν $T$ είναι η ταχύτητα της γραμμής και $D$ ο αριθμός των bits που αποστέλονται ανά πακέτο, τότε ισχύει:
\begin{align*}
	D & = 33bits     \\
	T & = 24Mbit/sec
\end{align*}

Από την δεύτερη εξίσωση προκύπτει ότι:
\begin{align*}
	T & = 24Mbit/sec                                                                  \\
	T & = \frac{24 \cdot 1024}{1} \frac{kbit}{sec} = 24 \cdot 1024^2 \frac{bits}{sec} \\
\end{align*}

Συνεπώς στα 35sec μπορούν να μεταφερθούν:
$$
	D_t = T \cdot 35sec = 24 \cdot 1024^2 \cdot 35 \ \frac{bits \cdot sec}{sec} = 880,803,840 bits
$$
όπου $D_t$ ο συνολικός όγκος δεδομένων που μπορεί να αποσταλεί σε $35sec$.

Τελικά, προκειμένου να δούμε πόσοι αριθμοί μπορούν να σταλούν αρκεί να
διαιρέσουμε το $D_t$ με το μέγεθος του ενός πακέτου($33bits$):

$$
	n = \frac{D_t}{D} = \frac{880,803,840 bits}{33 bits} = 26,691,025.\bar{45} \approx 26,691,025
$$

Θα αποσταλούν συνεπώς 26,691,025 αριθμοί των 32bit με parity check.

\section{Ερώτημα 2}
Κάθε μια από τις παρακάτω πράξεις μπορεί να είναι σωστή σε ένα ή περισσότερα αριθμητικά συστήματα. \\
Προσδιορίστε τις πιθανές βάσεις των αριθμητικών συστημάτων
% TODO: description for how I solved(make 1 and 2 similar to 3)

\textbf{Απάντηση: }
Σε όλα τα παρακάτω θεωρώ ότι: $a \in \mathbb{Z} \land a > 1$ όπου $a$ η βάση του αριθμητικού συστήματος γραμμένη σε βάση $10$.
Γνωρίζουμε ότι σε κάθε αριθμητικό σύστημα το κάθε ψηφίο ενός αριθμού αντιστοιχεί στον αριθμό των φορών που ο αριθμός περιέχει την δύναμη της βάσης που αντιστοιχεί στην στήλη του εκάστωτε ψηφίου.
Για παράδειγμα ο αριθμός $598$ στο δεκαδικό σύστημα, γράφεται και ως $5 \cdot 10^2 + 9 \cdot 10^1 + 8 \cdot 10^0$. Το αντίστοιχο ισχύει για όλα τα αριθμητικά συστήματα.
Αν λοιπόν $a$ η βάση του αριθμητικού συστήματος τότε για κάθε παράδειγμα ισχύει:
\subsection{Υποερώτημα α}
\begin{align*}
	\frac{51_a}{3_a}                & = 15_a               \\
	\frac{5_{10}a + 1_{10}}{3_{10}} & = 1_{10}a + 5_{10}   \\
	5_{10}a + 1_{10}                & = 3_{10}(a + 5_{10}) \\
	5_{10}a + 1_{10}                & = 3_{10}a + 15_{10}  \\
	2_{10}a                         & = 14_{10}            \\
	a                               & = 7_{10}
\end{align*}
Συνεπώς το $\frac{51}{3} = 15$ σημαίνει ότι οι πράξεις γίνονται στο αριθμητικό σύστημα με βάση το $7$.

\subsection{Υποερώτημα β}
\begin{align*}
	\frac{44_a}{4_a}                & = 11_a             \\
	\frac{4_{10}a + 4_{10}}{4_{10}} & = a + 1_{10}       \\
	4_{10}a + 4_{10}                & = 4_{10}a + 4_{10}
\end{align*}
Το παραπάνω ισχύει $\forall a \in \mathbb{Z}$. Συνεπώς σε κάθε αριθμητικό σύστημα που μπορεί να χρησιμοποιήσει το ψηφίο $4$ ισχύει ότι $\frac{44}{4} = 11$.
Για να μπορεί σε ένα σύστημα να αναπαρασταθεί το $4$ πρέπει η βάση του να είναι μεγαλύτερη του $4$. Συνεπώς το παραπάνω ισχύει για κάθε σύστημα από το πενταδικό και μετά.

\subsection{Υποερώτημα γ}
\begin{align*}
	\sqrt{51}        & = 6                               \\
	\sqrt{51_a}      & = 6_a = 6_{10}                    \\
	51_a             & = 6_{10}^2                        \\
	5_{10}a + 1_{10} & = 36_{10}                         \\
	5_{10}a          & = 35_{10}                         \\
	a                & = \frac{35_{10}}{5_{10}} = 7_{10}
\end{align*}
Παρατηρώ ότι επειδή το $6$ από μόνο του υπάρχει και έχει την ίδια τιμή σε όλα τα συστήματα με βάση $a \ge 7$ χρειάζεται να γίνουν κάποιες πράξεις σε διαφορετικά αριθμητικά συστήματα.
Γνωρίζουμε ότι $6_a = 6_{10}$ συνεπώς μπορούμε να αλλάξουμε την βάση του $6$. Ύστερα, μπορούμε να υψώσουμε και τις δύο πλευρές της εξίσωσης στο τετράγωνο. Από την μια πλευρά το αποτέλεσμα είναι σε βάση $a$ ενώ από την άλλη είναι βάση $10$.
Μπορούμε κανονικά να αναπτύξουμε το $6^2 = 36$ σε βάση $10$. Ύστερα αρκεί να γράψουμε και την αριστερή πλευρά σε βάση $10$. Τελικά με βασικές πράξεις μπορούμε να λύσουμε ως προς $a$ και βρίσκουμε $a = 7$.
Συνεπώς η ισότητα αυτή ισχύει στο επταδικό σύστημα. \\
Σημαντική παρατήρηση είναι ότι το $6$ μπορεί να αναπαρασταθεί στο επταδικό σύστημα με 1 ψηφίο. Αν αυτό δεν ίσχυε τότε θα έπρεπε να απορρίψουμε το αποτέλεσμα μας, μιας που η αρχική ισότητα χρησιμοποιεί το ψηφίο $6$.

\newpage

\section{Ερώτημα 3}

\subsection{Σχηματικό Διάγραμμα CMOS}
Σχεδιάστε το ολοκληρωμένο σχηματικό διάγραμμα CMOS διάγραμμα της συνάρτησης $Y=\overline{(A * B * C) + D}$ που θα περιέχει τόσο το pMOS όσο και το nMOS.

\textbf{Απάντηση: }\\
Αρχικά παρατηρούμε ότι η $Y$ είναι το συμπλήρωμα της συνάρτησης $\bar{Y} = (A * B * C) + D$.
Για να σχεδιάσουμε τα nMos θα χρησιμοποιήσουμε την $\bar{Y}$ αφού αυτά ελέγχουν πότε θα είναι λογικό 0 το αποτέλεσμα.
Η $\bar{Y}$ μπορεί να υλοποιηθεί με 4 nMos. Τρία από αυτά είναι συνδεδεμένα στη σειρά και στα gates τους είναι συνδεδεμένες οι είσοδοι $A, B$ και $C$.
Αυτό θα υλοποιήσει μια πύλη AND για της εισόδους αυτές. Αφού έχουμε και μια OR που συνδέει την $D$ με την έξοδο της προαναφερθήσας AND,
αρκεί απλά να συνδέσουμε την AND παράλληλα με άλλο ένα nMos το οποίο ως gate δέχεται το $D$.

Τώρα εφαρμόζοντας κανόνες De-Morgan στο $Y$ προκύπτει το εξής:
\begin{align*}
	Y & = \overline{(A * B * C) + D}              \\
	Y & = \overline{(A * B * C)} * \bar{D}        \\
	Y & = (\bar{A} + \bar{B} + \bar{C}) * \bar{D} \\
\end{align*}

Χρησιμοποιώντας την παραπάνω εξίσωση Boole μπορούμε να υλοποιήσουμε το pMos μέρος του κυκλώματος.
Αυτό θα αποτελείται από 3 pMos συνδεδεμένα παράλληλα, τα gates των οποίων θα είναι τα σήματα $A$, $B$ και $C$ το οποίο θα υλοποιεί μία OR πύλη.
Στη συνέχεια αρκεί το παραπάνω υποκύκλωμα να συνδεθεί στη σειρά με ένα pMos το gate του οποίου είναι συνδεδεμένο με την είσοδο $D$.

Βραχυκυκλώνοντας το pMos και το nMos κομμάτι του κυκλώματος προκύπτει η έξοδος $Y$. Ακολουθεί το \\
διάγραμμα που περιγράφεται παραπάνω:

\begin{center}
	\begin{circuitikz}
		\draw
		(3,0) node[vcc] (vcc) {$V_{cc}$}
		(3,-10) node[ground] (gnd) {}
		(3,-11) node[label] () {$GND$}
		(8, -5) node[label] (Y) {$Y$}
		% nmos part
		%% define mosfets
		(4, -8) node[nmos, xscale=-1] (Dnmos) {}
		(2, -9) node[nmos] (Cnmos) {}
		(2, -8) node[nmos] (Bnmos) {}
		(2, -7) node[nmos] (Anmos) {}
		%% define inputs
		(0, -7) node[label] (A) {$A$}
		(0, -8) node[label] (B) {$B$}
		(0, -9) node[label] (C) {$C$}
		(6, -8) node[label] (D) {$D$}
		%% define shorts
		(A) -- (Anmos.G)
		(B) -- (Bnmos.G)
		(C) -- (Cnmos.G)
		(D) -- (Dnmos.G)
		(gnd) -| (Cnmos.S)
		(gnd) -| (Dnmos.S)
		(Anmos.D) |- (3, -6)
		(Dnmos.D) |- (3, -6)
		(3, -6) to[short, -*] (3, -5)

		% pmos part
		%% define mosfets
		(3, -1) node[pmos] (Dpmos) {}
		(3, -3) node[pmos] (Bpmos) {}
		(0, -3) node[pmos] (Apmos) {}
		(6, -3) node[pmos] (Cpmos) {}
		%% define inputs
		(1, -1) node[label] (D) {$D$}
		(-2, -3) node[label] (A) {$A$}
		(1, -3) node[label] (B) {$B$}
		(4, -3) node[label] (C) {$C$}
		%% define shorts
		(vcc) -- (Dpmos.S)
		(Dpmos.D) |- (Bpmos.S)
		(Bpmos.S) -- (Apmos.S)
		(Bpmos.S) |- (Cpmos.S)
		(A) -- (Apmos.G)
		(B) -- (Bpmos.G)
		(C) -- (Cpmos.G)
		(Apmos.D) -- (Bpmos.D)
		(Cpmos.D) -- (Bpmos.D)
		(Bpmos.D) |- (Y)

		%% beautify
		(3, -6) |- (Y)
		;
	\end{circuitikz}
\end{center}

\newpage
\subsection{Πίνακας Αληθείας}
Γράψτε τον πίνακα αληθείας της συνάρτησης Y που υλοποιείται.

\textbf{Απάντηση: }
Σημαντική παρατήρηση είναι ότι αν η είσοδος $D$ είναι λογικό 1, όλες οι υπόλοιπες είσοδοι είναι αδιάφοροι όροι καθώς τότε η εξίσωση Boole παίρνει την μορφή $\overline{A * B * C + 1} = \bar{1} = 0$.
Για την απλοποίηση λοιπόν του πίνακα αληθείας μπορούμε να χρησιμοποιήσουμε x για τους αδιάφορους όρους.
\begin{center}
	\begin{tabular}{ | c | c | c | c | c | c | c | }
		\hline $D$ & $A$ & $B$ & $C$ & $ABC$ & $ABC + D$ & $Y$ \\
		\hline 0   & 0   & 0   & 0   & 0     & 0         & 1   \\
		0          & 0   & 0   & 1   & 0     & 0         & 1   \\
		0          & 0   & 1   & 0   & 0     & 0         & 1   \\
		0          & 0   & 1   & 1   & 0     & 0         & 1   \\
		0          & 1   & 0   & 0   & 0     & 0         & 1   \\
		0          & 1   & 0   & 1   & 0     & 0         & 1   \\
		0          & 1   & 1   & 0   & 0     & 0         & 1   \\
		0          & 1   & 1   & 1   & 1     & 1         & 0   \\
		1          & x   & x   & x   & x     & 1         & 0   \\
		\hline
	\end{tabular}
\end{center}

\subsection{Λογικό Διάγραμμα της $Y$}
Σχεδιάστε το διάγραμμα της συνάρτησης Υ με πύλες 2 εισόδων.

\textbf{Απάντηση: } Για να σχεδιάσουμε την $Y$ με πύλες δύο εισόδων πρέπει πρώτα να μετατρέψουμε όλες της πράξεις της σε δυαδικές με την χρήση της προσεταιριστικότητας των λογικών τελεστών.
Συνεπώς: $Y = \overline{(A * B * C) + D} = \overline{((A * B) * C) + D}$.
Το διάγραμμα της εξίσωσης στη μορφή αυτή είναι το παρακάτω:

\begin{center}
	\begin{circuitikz}
		\ctikzset{logic ports=ieee}
		\draw
		(3, 0) node[ieeestd and port] (aandb) {}
		(6, 0.6) node[ieeestd and port] (aandbandc) {}
		(9, 1.2) node[ieeestd or port] (orD) {}
		(11, 1.2) node[ieeestd not port] (not) {}
		(0, -0.3) node[label] (A) {$A$}
		(0, 0.3) node[label] (B) {$B$}
		(0, 0.9) node[label] (C) {$C$}
		(0, 1.5) node[label] (D) {$D$}
		(12, 1.2) node[label] (Y) {$Y$}

		(A) -- (aandb.in 2)
		(B) -- (aandb.in 1)
		(aandb.out) |-  (aandbandc.in 2)
		(C) -- (aandbandc.in 1)
		(aandbandc.out) |- (orD.in 2)
		(D) -- (orD.in 1)
		(orD.out) -- (not)
		(not.out) -- (Y)
		;
	\end{circuitikz}
\end{center}

Ωστόσο το παραπάνω μπορεί να απλοποιηθεί αν ενώσουμε την NOT με την OR για να δημιουργήσουμε μια πύλη NOR. Αυτό φαίνεται παρακάτω:

\begin{center}
	\begin{circuitikz}
		\ctikzset{logic ports=ieee}
		\draw
		(3, 0) node[ieeestd and port] (aandb) {}
		(6, 1) node[ieeestd and port] (aandbandc) {}
		(9, 2) node[ieeestd nor port] (orD) {}
		(aandb.in 2) ++ (-1, 0) node[label] (A) {$A$}
		(aandb.in 1) ++ (-1, 0) node[label] (B) {$B$}
		(aandbandc.in 1) ++ (-4, 0) node[label] (C) {$C$}
		(orD.in 1) ++ (-7, 0) node[label] (D) {$D$}
		(orD.out) ++ (1, 0) node[label] (Y) {$Y$}

		(A) -- (aandb.in 2)
		(B) -- (aandb.in 1)
		(aandb.out) |-  (aandbandc.in 2)
		(C) -- (aandbandc.in 1)
		(aandbandc.out) |- (orD.in 2)
		(D) -- (orD.in 1)
		(orD.out) -- (Y)
		;
	\end{circuitikz}
\end{center}

Αν ενδιαφερόμασταν για την ελαχιστοποίηση των MOSFET transistors που χρησιμοποιούνται, μπορούμε να την πετύχουμε με τη χρήση της μεθόδου της φυσαλίδας καθώς και την ιδιότητα $A * B = \overline{\bar{A} + \bar{B}}$ όπως φαίνεται παρακάτω.

\begin{center}
	\begin{circuitikz}
		\ctikzset{logic ports=ieee}
		\draw
		(3, 0) node[ieeestd and port] (aandb) {}
		(6, 1) node[ieeestd nor port] (aandbandc) {}
		(9, 2) node[ieeestd nor port] (orD) {}
		(aandb.in 2) ++ (-1, 0) node[label] (A) {$A$}
		(aandb.in 1) ++ (-1, 0) node[label] (B) {$B$}
		(aandbandc.in 1) ++ (-4, 0) node[label] (C) {$C$}
		(orD.in 1) ++ (-7, 0) node[label] (D) {$D$}
		(orD.out) ++ (1, 0) node[label] (Y) {$Y$}
		(aandbandc.bin 1) node[ocirc, left] {}
		(aandbandc.bin 2) node[ocirc, left] {}

		(A) -- (aandb.in 2)
		(B) -- (aandb.in 1)
		(aandb.out) |-  (aandbandc.in 2)
		(C) -- (aandbandc.in 1)
		(aandbandc.out) |- (orD.in 2)
		(D) -- (orD.in 1)
		(orD.out) -- (Y)
		;
	\end{circuitikz}
\end{center}

Το οποίο μπορούμε να απλοποιήσουμε κι άλλο και να προκύψει το παρακάτω:

\begin{center}
	\begin{circuitikz}
		\ctikzset{logic ports=ieee}
		\draw
		(3, 0) node[ieeestd nand port] (aandb) {}
		(6, 1) node[ieeestd nor port] (aandbandc) {}
		(9, 2) node[ieeestd nor port] (orD) {}
		(aandb.in 2) ++ (-1, 0) node[label] (A) {$A$}
		(aandb.in 1) ++ (-1, 0) node[label] (B) {$B$}
		(aandbandc.in 1) ++ (-4, 0) node[label] (C) {$C$}
		(orD.in 1) ++ (-7, 0) node[label] (D) {$D$}
		(orD.out) ++ (1, 0) node[label] (Y) {$Y$}
		(aandbandc.bin 1) node[ocirc, left] {}

		(A) -- (aandb.in 2)
		(B) -- (aandb.in 1)
		(aandb.out) |-  (aandbandc.in 2)
		(C) -- (aandbandc.in 1)
		(aandbandc.out) |- (orD.in 2)
		(D) -- (orD.in 1)
		(orD.out) -- (Y)
		;
	\end{circuitikz}
\end{center}

Στην αρχική υλοποίηση χρησιμοποιούσαμε τα παρακάτω:
\begin{center}
	\begin{tabular} { | c | c | }
		\hline Πύλη   & MOSFET transistors \\
		\hline NOR    & 4                  \\
		AND           & 6                  \\
		AND           & 6                  \\
		\hline Σύνολο & 16                 \\
		\hline
	\end{tabular}
\end{center}

Στην παραπάνω υλοποίηση ωστόσο χρησιμοποιούμε:
\begin{center}
	\begin{tabular} { | c | c | }
		\hline Πύλη   & MOSFET transistors \\
		\hline NOR    & 4                  \\
		NOR           & 4                  \\
		NAND          & 4                  \\
		NOT           & 2                  \\
		\hline Σύνολο & 14                 \\
		\hline
	\end{tabular}
\end{center}

\end{document}
