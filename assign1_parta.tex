\documentclass[draft]{article}
\usepackage{babel}
\usepackage{geometry}
\usepackage{fontspec}
\usepackage{amsmath}
\usepackage{amssymb}
\usepackage{circuitikz}
\setmainfont{NotoSans-Regular}
\geometry{margin=0.5in}
\selectlanguage{greek}

\author{Πλαστήρας Πέτρος}
\date{12 November 2024}
\title{Εργαστήριο Λογικής Σχεδίασης - Εργασία 1}
\begin{document}
\maketitle

\section{Μέρος Α}
\subsection{Ερώτημα 1}
Μία γραμμή internet έχει ταχύτητα 24Mbit/sec.\\
Πόσους μη προσημασμένους αριθμούς των 32bit μπορεί να μεταδώσει σε 35 sec, 
αν επιπλέον εφαρμόζει και parity check (περιττό) για κάθε αριθμό που στέλνει;\\
\textbf{Απάντηση: }\\
Παρατηρούμε αρχικά ότι για κάθε 32bit αριθμό, θα αποστέλονται στην πραγματικότητα 33bit αφού πρέπει να \\
αποσταλείκαι ένα επιπλέον bit για το parity check. Συνεπώς το ερώτημα είναι πόσα πακέτα των 33bit μπορούν να αποσταλούν μέσω της γραμμής σε 35 sec.
Αν $T$ είναι η ταχύτητα της γραμμής και $D$ ο αριθμός των bits που αποστέλονται ανά πακέτο, τότε ισχύει:
\begin{align*}
  D &= 33bits \\
  T &= 24Mbit/sec
\end{align*}

Από την δεύτερη εξίσωση προκύπτει ότι:
\begin{align*}
  T &= 24Mbit/sec \\
  T &= \frac{24 \cdot 1024}{1} \frac{kbit}{sec} = 24 \cdot 1024^2 \frac{bits}{sec}\\
\end{align*}

Συνεπώς στα 35sec μπορούν να μεταφερθούν:
$$
D_t = T \cdot 35sec = 24 \cdot 1024^2 \cdot 35 \ \frac{bits \cdot sec}{sec} = 880,803,840 bits
$$
όπου $D_t$ ο συνολικός όγκος δεδομένων που μπορεί να αποσταλεί σε $35sec$. 

Τελικά, προκειμένου να δούμε πόσοι αριθμοί μπορούν να σταλούν αρκεί να 
διαιρέσουμε το $D_t$ με το μέγεθος του ενός πακέτου($33bits$):

$$
n = \frac{D_t}{D} = \frac{880,803,840 bits}{33 bits} = 26,691,025.\bar{45} \approx 26,691,025
$$

Θα αποσταλούν συνεπώς 26,691,025 αριθμοί των 32bit με parity check.

\subsection{Ερώτημα 2}
Κάθε μια από τις παρακάτω πράξεις μπορεί να είναι σωστή σε ένα ή περισσότερα αριθμητικά συστήματα. \\
Προσδιορίστε τις πιθανές βάσεις των αριθμητικών συστημάτων 

\textbf{Απάντηση: }
Σε όλα τα παρακάτω θεωρώ ότι: $a \in \mathbb{Z} \land a > 1$ όπου $a$ η βάση του αριθμητικού συστήματος γραμμένη σε βάση $10$.
Γνωρίζουμε ότι σε κάθε αριθμητικό σύστημα το κάθε ψηφίο ενός αριθμού αντιστοιχεί στον αριθμό των φορών που ο αριθμός περιέχει την δύναμη της βάσης που αντιστοιχεί στην στήλη του εκάστωτε ψηφίου.
Για παράδειγμα ο αριθμός $598$ στο δεκαδικό σύστημα, γράφεται και ως $5 \cdot 10^2 + 9 \cdot 10^1 + 8 \cdot 10^0$. Το αντίστοιχο ισχύει για όλα τα αριθμητικά συστήματα.
Αν λοιπόν $a$ η βάση του αριθμητικού συστήματος τότε για κάθε παράδειγμα ισχύει:
\subsubsection{α}
\begin{align*}
  \frac{51_a}{3_a} &= 15_a \\
  \frac{5_{10}a + 1_{10}}{3_{10}} &= 1_{10}a + 5_{10} \\
  5a + 1 &= 3(a + 5) \\
  5a + 1 &= 3a + 15 \\
  2a &= 14 \\
  a &= 7
\end{align*}
Συνεπώς το $\frac{51}{3} = 15$ σημαίνει ότι οι πράξεις γίνονται στο αριθμητικό σύστημα με βάση το $7$.

\subsubsection{β}
\begin{align*}
  \frac{44_a}{4_a} &= 11_a\\ 
  \frac{4_{10}a + 4_{10}}{4_{10}} &= a + 1_{10}\\
  4a + 4 &= 4a + 4
\end{align*}
Το παραπάνω ισχύει $\forall a \in \mathbb{Z}$. Συνεπώς σε κάθε αριθμητικό σύστημα που μπορεί να χρησιμοποιήσει το ψηφίο $4$ ισχύει ότι $\frac{44}{4} = 11$.
Για να μπορεί σε ένα σύστημα να αναπαρασταθεί το $4$ πρέπει η βάση του να είναι μεγαλύτερη του $4$. Συνεπώς το παραπάνω ισχύει για κάθε σύστημα μετά το πενταδικό.

\subsubsection{γ}
\begin{align*}
  \sqrt{51} &= 6 \\
  \sqrt{51_a} &= 6_a = 6_{10} \\
  51_a &= 6_{10}^2 \\
  5_{10}a + 1_{10} &= 36_{10} \\
  5_{10}a &= 35_{10} \\
  a &= \frac{35_{10}}{5_{10}} = 7_{10}
\end{align*}
Παρατηρώ ότι επειδή το $6$ από μόνο του υπάρχει και έχει την ίδια τιμή σε όλα τα συστήματα με βάση $a \ge 7$ χρειάζεται να γίνουν κάποιες πράξεις σε διαφορετικά αριθμητικά συστήματα.
Γνωρίζουμε ότι $6_a = 6_{10}$ συνεπώς μπορούμε να αλλάξουμε την βάση του $6$. Ύστερα, μπορούμε να υψώσουμε και τις δύο πλευρές της εξίσωσης στο τετράγωνο. Από την μια πλευρά το αποτέλεσμα είναι σε βάση $a$ ενώ από την άλλη είναι βάση $10$.
Μπορούμε κανονικά να αναπτύξουμε το $6^2 = 36$ σε βάση $10$. Ύστερα αρκεί να γράψουμε και την αριστερή πλευρά σε βάση $10$. Τελικά με βασικές πράξεις μπορούμε να λύσουμε ως προς $a$ και βρίσκουμε $a = 7$.
Συνεπώς η ισότητα αυτή ισχύει στο επταδικό σύστημα. \\ 
Σημαντική παρατήρηση είναι ότι το $6$ μπορεί να αναπαρασταθεί στο επταδικό σύστημα με 1 ψηφίο. Αν αυτό δεν ίσχυε τότε θα έπρεπε να απορρίψουμε το αποτέλεσμα μας, μιας που η αρχική ισότητα χρησιμοποιεί το ψηφίο $6$.

\newpage

\subsection{Ερώτημα 3}
Σχεδιάστε το ολοκληρωμένο σχηματικό διάγραμμα CMOS διάγραμμα της συνάρτησης $Y=\overline{(A * B * C) + D}$ που θα περιέχει τόσο το pMOS όσο και το nMOS.\\
\textbf{Απάντηση: } \\
\begin{center}
  \begin{circuitikz}
    \draw 
    (3,0) node[vcc] (vcc) {$V_{cc}$}
    (3,-10) node[ground] (gnd) {}
    (3,-11) node[label] () {$GND$}
    (8, -5) node[label] (Y) {$Y$}
    % nmos part
    %% define mosfets
    (4, -8) node[nmos, xscale=-1] (Dnmos) {}
    (2, -9) node[nmos] (Cnmos) {}
    (2, -8) node[nmos] (Bnmos) {}
    (2, -7) node[nmos] (Anmos) {}
    %% define inputs
    (0, -7) node[label] (A) {$A$}
    (0, -8) node[label] (B) {$B$}
    (0, -9) node[label] (C) {$C$}
    (6, -8) node[label] (D) {$D$}
    %% define shorts
    (A) -- (Anmos.G)
    (B) -- (Bnmos.G)
    (C) -- (Cnmos.G)
    (D) -- (Dnmos.G)
    (gnd) -| (Cnmos.S)
    (gnd) -| (Dnmos.S)
    (Anmos.D) |- (3, -6)
    (Dnmos.D) |- (3, -6)

    % pmos part
    %% define mosfets
    (3, -1) node[pmos] (Dpmos) {}
    (3, -3) node[pmos] (Bpmos) {}
    (0, -3) node[pmos] (Apmos) {}
    (6, -3) node[pmos] (Cpmos) {}
    %% define inputs
    (1, -1) node[label] (D) {$D$}
    (-2, -3) node[label] (A) {$A$}
    (1, -3) node[label] (B) {$B$}
    (4, -3) node[label] (C) {$C$}
    %% define shorts
    (vcc) -- (Dpmos.S)
    (Dpmos.D) |- (Bpmos.S)
    (Bpmos.S) -- (Apmos.S)
    (Bpmos.S) |- (Cpmos.S)
    (A) -- (Apmos.G)
    (B) -- (Bpmos.G)
    (C) -- (Cpmos.G)
    (Apmos.D) -- (Bpmos.D)
    (Cpmos.D) -- (Bpmos.D)
    (Bpmos.D) |- (Y)

    %% beautify
    (3, -6) |- (Y)
    ;
  \end{circuitikz}
\end{center}
\end{document}
