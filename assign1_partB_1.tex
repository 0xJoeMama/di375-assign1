\documentclass[draft]{article}

\usepackage{geometry}
\usepackage{fontspec}
\usepackage{amsmath}
\usepackage{amssymb}
\usepackage{circuitikz}
\usepackage{minted2}

\setmainfont{NotoSans}
\geometry{margin=0.5in}

\begin{document}

\author{Πλαστήρας Πέτρος}
\title{Εργαστήριο Λογικής Σχεδίασης - Εργασία 1}
\date{25 Νοεμβρίου 2024}
\maketitle

\section{Μέρος Β.1}
Σχεδιάστε ένα σχηματικό διάγραμμα του αρχικού κυκλώματος που περιγράφεται από 
τον ακόλουθο κώδικα VHDL. Στη συνέχεια, απλουστεύστε το διάγραμμα ώστε να 
περιέχει όσο το δυνατόν λιγότερες πύλες. Τέλος, δημιουργήστε τον πίνακα αληθείας.

\inputminted{vhdl}{assign1_partB_1.vhdl}

\textbf{Απάντηση: } Αρχικά παρατηρούμε από τον ορισμό της οντότητας ότι το 
κύκλωμα έχει 3 εισόδους(a, b και c) και 2 εξόδους(y, z).
Από την αρχιτεκτονική της οντότητας μπορούμε να βρούμε τις εξισώσεις Boole των εξόδων.
Συγκεκριμένα, παρατηρώντας τις τιμές που παίρνουν τα σήματα y και z προκύπτει το διάγραμμα \ref{fig:init}.

\begin{align*}
  Y &= A * \bar{B} * \bar{C} + A * B * C + \bar{A} \\
  Z &= A * \bar{B} + \bar{A} * B
\end{align*}

Προκειμένου να απλοποιήσουμε το διάγραμμα, θα χρησιμοποιήσουμε τις εξισώσεις και τα θεωρήματα της\\
άλγεβρας Boole. Για την απλοποίηση του $Y$ αρχικά θα χρησιμοποιήσουμε την επιμεριστικότητα του $*$:
\begin{align*}
  Y &= A * \bar{B} * \bar{C} + A * B * C + \bar{A} \\
  Y &= A * (\bar{B} * \bar{C} + B * C) + \bar{A} 
\end{align*}

Στη συνέχεια μπορούμε να χρησιμοποιήσουμε το θεώρημα της κάλυψης:
\begin{align*}
  Y &= A * (\bar{B} * \bar{C} + B * C) + \bar{A} \\
  Y &= \bar{B} * \bar{C} + B * C + \bar{A} 
\end{align*}

Αρκεί τέλος να απλοποιήσουμε την έκφραση $\bar{B} * \bar{C} + B * C$. Για να το κάνουμε αυτό μπορούμε να χρησιμοποιήσουμε πίνακα αληθείας:
\begin{center}
  \begin{tabular} {|c|c|c|c|}
    \hline \rule{0pt}{11pt}$B$ & $C$ & $\bar{B} * \bar{C} + B * C$ & $\overline{B \oplus C}$\\
    \hline 0 & 0 & 1 & 1\\
           0 & 1 & 0 & 0\\
           1 & 0 & 0 & 0\\
           1 & 1 & 1 & 1\\
    \hline
  \end{tabular}
\end{center}

Παρατηρούμε ότι η στήλη του $\bar{B} * \bar{C} + B * C$ είναι πανομοιότυπη με αυτή του XNOR. Συνεπώς το $Y$ μπορεί να γραφτεί και ως εξής:
\begin{align*}
  Y &= \bar{B} * \bar{C} + B * C + \bar{A} \\
  Y &= \overline{B \oplus C} + \bar{A}\\
\end{align*}

Για την απλοποίηση του $Z$ αντίστοιχα θα χρησιμοποιήσουμε τον πίνακα αληθείας:
\begin{center}
  \begin{tabular} {|c|c|c|c|}
    \hline \rule{0pt}{11pt}$A$ & $B$ & $Z = \bar{A} * B + A * \bar{B}$ & $A \oplus B$\\
    \hline 0 & 0 & 0 & 0\\
           0 & 1 & 1 & 1\\
           1 & 0 & 1 & 1\\
           1 & 1 & 0 & 0\\
    \hline
  \end{tabular}
\end{center}

Γίνεται λοιπόν εμφανές ότι:
\begin{align*}
  Z &= A * \bar{B} + \bar{A} * B\\
  Z &= A \oplus B
\end{align*}

Σχηματικά μπορούμε να υλοποιήσουμε τις εξισώσεις αυτές με τον παρακάτω τρόπο:
\begin{figure}
  \begin{center}
      \begin{circuitikz}
        \ctikzset{logic ports=ieee}
        \draw
        (0, 0) node[ocirc, label=$C$, label distance = 2pt] (C) {}
        (1, 0) node[ocirc, label=$B$, label distance = 2pt] (B) {}
        (2, 0) node[ocirc, label=$A$, label distance = 2pt] (A) {}
        (8, -9) node[rotate=270, ieeestd or port, number inputs = 3] (res) {}
        (res.bin 1) node[ocirc, above] (notIn) {}
        (res.in 2) ++ (-4, 5) node[ieeestd and port, number inputs = 3] (abc) {}
        (res.in 2) ++ (-4, 7) node[ieeestd and port, number inputs = 3] (anotbnotc) {}
        (anotbnotc.bin 2) node[ocirc, left] (notB) {}
        (anotbnotc.bin 1) node[ocirc, left] (notC) {}
        (res.out) node[ocirc] (Y) {}
        (Y.south) ++ (0, -0.2) node[label] {$Y$}

        (6, -9) node[rotate=270, ieeestd or port] (zOut) {}
        (zOut.out) node[ocirc] (Z) {}
        (abc) ++ (0, -2) node[ieeestd and port] (aandnotb) {}
        (aandnotb.bin 2) node[ocirc, left] (notBAg) {}
        (aandnotb) ++ (0, -2) node[ieeestd and port] (notaandb) {}
        (notaandb.bin 1) node[ocirc, left] (notAAg) {}
        (Z.south) ++ (0, -0.2) node[label] {$Z$}

      
        (A) |- (notAAg)
        (B) |- (notaandb.in 2)
        (notaandb.out) -| (zOut.in 2)
        (aandnotb.out) -| (zOut.in 1)
        (B) |- (notBAg)
        (A) |- (aandnotb.in 1)
        (B) |- (abc.in 2)
        (C) |- (abc.in 1)
        (A) |- (abc.in 3)
        (B) |- (anotbnotc.in 2)
        (C) |- (anotbnotc.in 1)
        (A) |- (anotbnotc.in 3)
        (anotbnotc.out) -| (res.in 2)
        (abc.out) -| (res.in 3)
        (A) -| (notIn)
        ;
      \end{circuitikz}

  \end{center}
  \caption{Σχηματικό διάγραμμα της αρχικής εξίσωσης}
  \label{fig:init}
\end{figure}

\begin{figure}
  \begin{center}
    \begin{circuitikz}
      \draw
        (0, 0) node[ieeestd xor port] (xor) {}
        (xor.in 1) ++ (-2, 0) node[ocirc, label=$A$, label distance = 2pt] (A) {}
        (xor.in 2) ++ (-2, 0) node[ocirc, label=$B$, label distance = 2pt] (B) {}
        (B) ++ (0, -1) node[ocirc, label=$C$, label distance = 2pt] (C) {}
        (xor) ++ (0, -2) node[ieeestd xnor port] (xnor) {}
        (xnor.out) ++ (2, 1) node[ieeestd or port] (or) {}
        (or.bin 1) node[ocirc, left] (notA) {}

        (or.out) node[ocirc, label=$Y$] (Y) {}
        (Y) ++ (0, 1) node[ocirc, label=$Z$] (Z) {}

        (xor.out) |- (Z)
        (A) ++ (1.5, 0) |- (or.in 1)
        (xnor.out) |- (or.in 2)
        (xor.in 1) |- (A)
        (xor.in 2) |- (B)
        (C) |- (xnor.in 2)
        (B) ++ (1, 0) |- (xnor.in 1)
        ;
    \end{circuitikz}
  \end{center}
  \caption{Σχηματικό διάγραμμα απλοποιημένης εξίσωσης}
\end{figure}

\begin{center}
  \begin{tabular}{|c|c|c|c|c|}
    \hline \rule{0pt}{11pt} $A$ & $B$ & $C$ & $Y = \overline{B \oplus C} + \bar{A}$ & $Z = A \oplus B$\\
    \hline 0 & 0 & 0 & 1 & 0 \\
           0 & 0 & 1 & 1 & 0 \\
           0 & 1 & 0 & 1 & 1 \\
           0 & 1 & 1 & 1 & 1 \\
           1 & 0 & 0 & 1 & 1 \\
           1 & 0 & 1 & 0 & 1 \\
           1 & 1 & 0 & 0 & 0 \\
           1 & 1 & 1 & 1 & 0 \\
      \hline
  \end{tabular}
\end{center}
\end{document}
