\documentclass[]{article}

\usepackage{geometry}
\usepackage{fontspec}
\usepackage{amsmath}
\usepackage{amssymb}
\usepackage{circuitikz}
\usepackage{minted2}

\setmainfont{NotoSans}
\geometry{margin=0.5in}

\begin{document}

\author{Πλαστήρας Πέτρος}
\title{Εργαστήριο Λογικής Σχεδίασης - Εργασία 1 - Μερός Β.1}
\date{25 Νοεμβρίου 2024}
\maketitle

Σχεδιάστε ένα σχηματικό διάγραμμα του αρχικού κυκλώματος που
περιγράφεται από τον ακόλουθο κώδικα VHDL, με πύλες 2 εισόδων. Στη συνέχεια,
απλουστεύστε (αν γίνεται) το διάγραμμα ώστε να περιέχει όσο το δυνατόν λιγότερες
πύλες (πάντα 2 εισόδων). Τέλος, δημιουργήστε τον πίνακα αληθείας.

\inputminted{vhdl}{./b1_prompt.vhdl}

\section{Εξισώσεις Boole}
Αρχικά παρατηρούμε από τον ορισμό της οντότητας ότι το κύκλωμα έχει 3 εισόδους(a, b και c) και 2 εξόδους(y, z).
Από την αρχιτεκτονική της οντότητας μπορούμε να βρούμε τις εξισώσεις Boole των εξόδων.
Συγκεκριμένα, παρατη- ρώντας τις τιμές που παίρνουν τα σήματα y και z προκύπτουν οι παρακάτω εξισώσεις:
\begin{align*}
	Y & = ((A * \bar{B}) * \bar{C}) + ((A * B) * C) + \bar{A} \\
	  & = A * \bar{B} * \bar{C} + A * B * C + \bar{A}         \\
	Z & = A * \bar{B} + \bar{A} * B
\end{align*}

Προκειμένου να απλοποιήσουμε το διάγραμμα, θα χρησιμοποιήσουμε τις εξισώσεις και τα θεωρήματα της\\
άλγεβρας Boole. Για την απλοποίηση του $Y$ αρχικά θα χρησιμοποιήσουμε την επιμεριστικότητα του $*$:
\begin{align*}
	Y & = A * \bar{B} * \bar{C} + A * B * C + \bar{A} \\
	Y & = A * (\bar{B} * \bar{C} + B * C) + \bar{A}
\end{align*}

Στη συνέχεια μπορούμε να χρησιμοποιήσουμε το θεώρημα της επιμεριστικότητας:
\begin{align*}
	Y & = (A + \bar{A}) * (\bar{B} * \bar{C} + B * C + \bar{A}) \\
	  & = 1 * (\bar{B} * \bar{C} + B * C + \bar{A})             \\
	  & = \bar{B} * \bar{C} + B * C + \bar{A}                   \\
\end{align*}

Αρκεί τέλος να απλοποιήσουμε την έκφραση $\bar{B} * \bar{C} + B * C$. Για να το κάνουμε αυτό μπορούμε να χρησιμοποιήσουμε πίνακα αληθείας:
\begin{center}
	\begin{tabular} {|c|c|c|c|}
		\hline \rule{0pt}{11pt}$B$ & $C$ & $\bar{B} * \bar{C} + B * C$ & $\overline{B \oplus C}$ \\
		\hline 0                   & 0   & 1                           & 1                       \\
		0                          & 1   & 0                           & 0                       \\
		1                          & 0   & 0                           & 0                       \\
		1                          & 1   & 1                           & 1                       \\
		\hline
	\end{tabular}
\end{center}

Παρατηρούμε ότι η στήλη του $\bar{B} * \bar{C} + B * C$ είναι πανομοιότυπη με αυτή του XNOR. Συνεπώς το $Y$ μπορεί να γραφτεί και ως εξής:
\begin{align*}
	Y & = \bar{B} * \bar{C} + B * C + \bar{A} \\
	  & = \overline{B \oplus C} + \bar{A}     \\
\end{align*}

Μπορούμε τέλος χρησιμοποιώντας κανόνες De Morgan να απλοποιήσουμε ακόμη περισσότερο την συνάρτηση $Y$:
\begin{align*}
	Y & = \overline{B \oplus C} + \bar{A} \\
    & = \overline{A * (B \oplus C)}
\end{align*}

Αυτό μειώνει τον αριθμό των πυλών που χρησιμοποιούμε κατά 1(ο αντιστροφέας του $A$).

Για την απλοποίηση του $Z$ αντίστοιχα θα χρησιμοποιήσουμε τον πίνακα αληθείας:
\begin{center}
	\begin{tabular} {|c|c|c|c|}
		\hline \rule{0pt}{11pt}$A$ & $B$ & $Z = \bar{A} * B + A * \bar{B}$ & $A \oplus B$ \\
		\hline 0                   & 0   & 0                               & 0            \\
		0                          & 1   & 1                               & 1            \\
		1                          & 0   & 1                               & 1            \\
		1                          & 1   & 0                               & 0            \\
		\hline
	\end{tabular}
\end{center}

Γίνεται λοιπόν εμφανές ότι:
\begin{align*}
	Z & = A * \bar{B} + \bar{A} * B \\
	  & = A \oplus B                \\
\end{align*}

\section{Λογικά Διαγράμματα}
Η αρχική μη απλοποιημένη εξίσωση αντιστοιχεί στο παρακάτω λογικό κύκλωμα:
% TODO: replace with AND-2 and OR-2 because vasilopoulos changed the prompt last moment
\begin{center}
	\begin{circuitikz}
		\ctikzset{logic ports=ieee}
		\draw
		(-2, 0) node[ocirc, label=$C$, label distance = 2pt] (C) {}
		(0, 0) node[ocirc, label=$B$, label distance = 2pt] (B) {}
		(2, 0) node[ocirc, label=$A$, label distance = 2pt] (A) {}

		(A) ++ (1, -1) node[rotate=270, ieeestd not port] (notA) {}
		(B) ++ (1, -1) node[rotate=270, ieeestd not port] (notB) {}
		(C) ++ (1, -1) node[rotate=270, ieeestd not port] (notC) {}

		(A) |- (notA.in)
		(B) |- (notB.in)
		(C) |- (notC.in)

		(10, -10) node[rotate=270, ieeestd or port] (res) {}

		% Y output and label
		(res.out) node[ocirc] (Y) {}
		(Y.south) ++ (0, -0.2) node[label] {$Y$}

		% Z output and label
		(6, -10) node[rotate=270, ieeestd or port] (zOut) {}
		(zOut.out) node[ocirc] (Z) {}
		(Z.south) ++ (0, -0.2) node[label] {$Z$}

		% output of parts of Z and OR
		(zOut.in 1) ++ (-2, 1) node[ieeestd and port] (BandnotA) {}
		(BandnotA) ++ (0, 2) node[ieeestd and port] (AandnotB) {}
		(AandnotB.out) -| (zOut.in 1)
		(BandnotA.out) -| (zOut.in 2)

		% first part of Y
		(AandnotB.out) ++ (2, 1) node[ieeestd and port] (ABbarCbar) {}
		(notC) |- (ABbarCbar.in 1)
		(AandnotB.out) -| (ABbarCbar.in 2)
		(ABbarCbar) ++ (-3, 1) node[ieeestd and port] (AB) {}
		(ABbarCbar) ++ (0, 2) node[ieeestd and port] (ABC) {}
		(ABbarCbar.out) ++ (1, -2) node[rotate=270, ieeestd or port] (halfY) {}

		(notA.out) |- (BandnotA.in 1)
		(notB.out) |- (AandnotB.in 1)
		(B) |- (BandnotA.in 2)
		(A) |- (AandnotB.in 2)
		(A) |- (AB.in 1)
		(B) |- (AB.in 2)
		(C) |- (ABC.in 1)
		(halfY.out) -| (res.in 2)
		(ABbarCbar.out) -| (halfY.in 2)
		(ABC.out) -| (halfY.in 1)
		(notA.out) -| (res.in 1)
		(AB.out) -| (ABC.in 2)
		;
	\end{circuitikz}
\end{center}

Αντίστοιχα η τελική μορφή των εξισώσεων αντιστοιχεί στο παρακάτω λογικό κύκλωμα:
\begin{center}
	\begin{circuitikz}
		\draw
		(0, 0) node[ieeestd xor port] (xor) {}
		(xor.in 1) ++ (-2, 0) node[ocirc, label=$A$, label distance = 2pt] (A) {}
		(xor.in 2) ++ (-2, 0) node[ocirc, label=$B$, label distance = 2pt] (B) {}
		(B) ++ (0, -1) node[ocirc, label=$C$, label distance = 2pt] (C) {}
		(xor) ++ (0, -2) node[ieeestd xnor port] (xnor) {}
		(xnor.out) ++ (2, 1) node[ieeestd nand port] (or) {}

    (or.out) node[ocirc, label=$Y$] (Y) {}
		(Y) ++ (0, 1) node[ocirc, label=$Z$] (Z) {}

		(xor.out) |- (Z)
		(A) ++ (1.5, 0) |- (or.in 1)
		(xnor.out) |- (or.in 2)
		(xor.in 1) |- (A)
		(xor.in 2) |- (B)
		(C) |- (xnor.in 2)
		(B) ++ (1, 0) |- (xnor.in 1)
		;
	\end{circuitikz}
\end{center}

Παρατηρούμε την σημαντική μείωση του υλικού που χρησιμοποιείται για την υλοποίηση του κυκλώματος, η οποία οφείλεται στην απλοποίηση των εξισώσεων Boole που αντιστοιχούν σε αυτό.

\section{Πίνακας Αληθείας}
Χρησιμοποιώντας τις εξισώσεις Boole μπορούμε να φτιάξουμε και τον πίνακα αληθείας των συναρτήσεων:
\begin{center}
	\begin{tabular}{|c|c|c|c|c|c|c|}
		\hline \rule{0pt}{11pt} $A$ & $B$ & $C$ & $\bar{A}$ & $\overline{B \oplus C}$ & $Y = \overline{B \oplus C} + \bar{A}$ & $Z = A \oplus B$ \\
		\hline 0                    & 0   & 0   & 1         & 1                       & 1                                     & 0                \\
		0                           & 0   & 1   & 1         & 0                       & 1                                     & 0                \\
		0                           & 1   & 0   & 1         & 0                       & 1                                     & 1                \\
		0                           & 1   & 1   & 1         & 1                       & 1                                     & 1                \\
		1                           & 0   & 0   & 0         & 1                       & 1                                     & 1                \\
		1                           & 0   & 1   & 0         & 0                       & 0                                     & 1                \\
		1                           & 1   & 0   & 0         & 0                       & 0                                     & 0                \\
		1                           & 1   & 1   & 0         & 1                       & 1                                     & 0                \\
		\hline
	\end{tabular}
\end{center}
\end{document}
